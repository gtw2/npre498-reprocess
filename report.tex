% use the answers clause to get answers to print; otherwise leave it out.
\documentclass[12pts,answers]{exam}
\RequirePackage{amssymb, amsfonts, amsmath, latexsym, verbatim, xspace, setspace}
\usepackage{graphicx}

% By default LaTeX uses large margins.  This doesn't work well on exams; problems
% end up in the "middle" of the page, reducing the amount of space for students
% to work on them.
\usepackage[margin=1in]{geometry}
\usepackage{enumerate}
\usepackage[hidelinks]{hyperref}

% Here's where you edit the Class, Exam, Date, etc.
\newcommand{\class}{NPRE 498}
\newcommand{\term}{Fall 2018}
\newcommand{\assignment}{Final}
\newcommand{\duedate}{2018.12.12}
%\newcommand{\timelimit}{50 Minutes}

\newcommand{\nth}{n\ensuremath{^{\text{th}}} }
\newcommand{\ve}[1]{\ensuremath{\mathbf{#1}}}
\newcommand{\Macro}{\ensuremath{\Sigma}}
\newcommand{\vOmega}{\ensuremath{\hat{\Omega}}}

% For an exam, single spacing is most appropriate
%\singlespacing
\onehalfspacing
% \doublespacing

% For an exam, we generally want to turn off paragraph indentation
\parindent 0ex

%\unframedsolutions

\begin{document} 

% These commands set up the running header on the top of the exam pages
\pagestyle{head}
\firstpageheader{}{}{}
\runningheader{\class}{\assignment\ - Page \thepage\ of \numpages}{Due \duedate}
\runningheadrule

\class \hfill \term \\
\assignment \hfill Due \duedate\\
\rule[1ex]{\textwidth}{.1pt}
%\hrulefill

\section{Introduction}
Pyroprocessing is an electrochemical separation method used primarily for metallic fast reactor fuel.
This reprocessing technique uses molten salt, which differs depending on the facility, to provide a medium for current to travel across.
Molten salt such as LiCl has a broader stability range comparative to water, allowing high potentials to be used for separation.
Traditionally, separation would be conducted in a nitric acid which uses water as its medium.
Water, however, has a significantly lower stability compared to molten salt.
This becomes a problem when considering higher elements such as lanthanides and actinides.
Controlling the oxidation states of these elements often requires potentials outside the stability of water.
Hence, Pyroprocessing was born to improve nonproliferation and reprocessing capabilities.
\\ \\
In addition to the additional redox control of higher elements, we also co-extract materials of interest such that they cannot easily be refined for weapons.
This is done through the electrorefining and electrowinning stages by separating a pure Uranium stream as well as a Uranium/Transuranic mix stream. 
The U/TRU can then be readily used for fuel fabrication while maintaining proliferation resistance.
\section{Theory and Mechanisms}
\subsection{Electrochemical Separations}

\subsection{Molten Salt}

\section{Process}
\subsection{Voloxidation}

\subsection{Electroreduction}

\subsection{Electrorefining}

\subsection{Electrowinning}
\section{Summary}


\bibliography{ieeetr}
\end{document}